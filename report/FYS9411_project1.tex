\documentclass[pdftex,10pt,b5paper,twoside]{book}
\usepackage[lmargin=25mm,rmargin=25mm,tmargin=27mm,bmargin=30mm]{geometry}

\usepackage{setspace}
\usepackage{graphicx}
\usepackage{amssymb}
\usepackage{mathrsfs}
\usepackage{amsthm}
\usepackage{amsmath}
\DeclareMathOperator{\arsinh}{arsinh}
\DeclareMathOperator{\arcosh}{arcosh}
\DeclareMathOperator{\sirh}{sirh}
\usepackage{bm}
\usepackage{relsize}
\usepackage{breqn}
\DeclareMathOperator{\consum}{\mathlarger{\mathlarger{\mathlarger{\ddagger}}}}
\usepackage{courier}
\usepackage{physics}
\usepackage{color}
\usepackage[Lenny]{fncychap}
\usepackage[pdftex,bookmarks=true]{hyperref}
\usepackage[pdftex]{hyperref}
\hypersetup{
    colorlinks,%
    citecolor=black,%
    filecolor=black,%
    linkcolor=black,%
    urlcolor=black
}
\usepackage[font=small,labelfont=bf]{caption}
\usepackage{fancyhdr}
\usepackage{times}
%\usepackage[intoc]{nomencl}
%\renewcommand{\nomname}{List of Abbreviations}
%\makenomenclature
\usepackage{natbib}
\usepackage{float}
%\floatstyle{boxed} 
\restylefloat{figure}

\usepackage{listings}
\definecolor{codeyellow}{rgb}{0.8,0.5,0.0}
\definecolor{codeblue}{rgb}{0.0,0.0,0.6}
\definecolor{codered}{rgb}{0.8,0,0.0}
\definecolor{codegreen}{rgb}{0.0,0.5,0.0}
\definecolor{codewhite}{rgb}{1.0,1.0,1.0}
\definecolor{codeblack}{rgb}{0.0,0.0,0.0}

\lstdefinestyle{mystyle}{
    backgroundcolor=\color{codewhite},   
    commentstyle=\color{codeyellow}\textbf,
    keywordstyle=\color{codeblue}\textbf,
    numberstyle=\tiny\color{codeblack},
    stringstyle=\color{codered},
    basicstyle=\ttfamily\tiny,
    breakatwhitespace=false,         
    breaklines=true,
    postbreak=\mbox{\textcolor{codegreen}{$\hookrightarrow$}\space},               
    captionpos=b,                    
    keepspaces=true,                 
    numbers=left,                    
    numbersep=5pt,                  
    showspaces=false,                
    showstringspaces=false,
    showtabs=false,                 
    tabsize=2
}
 
\lstset{style=mystyle}


% ShareLaTeX does not support glossaries now. Sorry...
%\usepackage[number=none]{glossary}
%\makeglossary
%\newglossarytype[abr]{abbr}{abt}{abl}
%\newglossarytype[alg]{acronyms}{acr}{acn}
%\newcommand{\abbrname}{Abbreviations} 
%\newcommand{\shortabbrname}{Abbreviations}
%%\makeabbr
\newcommand{\HRule}{\rule{\linewidth}{0.5mm}}

\renewcommand*\contentsname{Table of Contents}

% New definition of square root:
% it renames \sqrt as \oldsqrt
\let\oldsqrt\sqrt
% it defines the new \sqrt in terms of the old one
\def\sqrt{\mathpalette\DHLhksqrt}
\def\DHLhksqrt#1#2{%
\setbox0=\hbox{$#1\oldsqrt{#2\,}$}\dimen0=\ht0
\advance\dimen0-0.2\ht0
\setbox2=\hbox{\vrule height\ht0 depth -\dimen0}%
{\box0\lower0.4pt\box2}}

\pagestyle{fancy}
\fancyhf{}
\renewcommand{\chaptermark}[1]{\markboth{\chaptername\ \thechapter.\ #1}{}}
\renewcommand{\sectionmark}[1]{\markright{\thesection\ #1}}
\renewcommand{\headrulewidth}{0.1ex}
\renewcommand{\footrulewidth}{0.1ex}
\fancypagestyle{plain}{\fancyhf{}\fancyfoot[LE,RO]{\thepage}\renewcommand{\headrulewidth}{0ex}}

\begin{document}

En $D$-dimensjonal kvantebrønn med $N$ partikler har Hamilton-operator

\begin{equation}
\mathbf{H}[\vec{R}] = \sum\limits_i^N \left(U[\vec{r}_i] -\frac{\hbar^2}{2m}\vec{\nabla}_i^2 \right) + \sum\limits_i^N\sum\limits_{j > i}^N V[\vec{r}_i,\vec{r}_j]
\end{equation}
i posisjonsbasis, med
\begin{align}
U[\vec{r}_i] &= \frac{m\omega^2}{2}\left(x_i^2+y_i^2+\lambda^2z_i^2\right), \\
V[\vec{r}_i,\vec{r}_j] &= V[\Delta{r}_{ij}] =\begin{cases} \infty & \text{for $\Delta{r}_{ij} < a$} \\
0 & \text{for $\Delta{r}_{ij} \geq a$} \end{cases},
\end{align}
der $U$ altså er kvantebrønn-feltet mens $V$ beskriver en hard veksling mellom partiklene som hindrer at de overlapper. Her er $\vec{r}_i$ og $\vec{\nabla}_i$ henholdsvis posisjonen og den posisjonsderiverte til hver partikkel $i$, mens $\vec{R}$ er den samla posisjonsvektoren for hele systemet, $a$ er en karakteristisk radius for partiklene, $m$ er partiklenes masse, $\omega$ er kvantebrønnstyrken og $\lambda$ er en parameter som eventuelt gjør kvantebrønnen elliptisk. Videre er
\begin{align}
\Delta\vec{r}_{ij} &= \vec{r}_i-\vec{r}_j \\
\Longrightarrow \Delta{r}_{ij} &= \sqrt{\left(\vec{r}_i-\vec{r}_j\right)^2}
\end{align}
avstandsvektoren mellom partiklene $i$ og $j$.

Innfør først $\sqrt{\frac{\hbar}{m\omega}}$ som lengdemål og $\hbar\omega$ som energimål ved å spalte ut
\begin{align}
a &= \sqrt{\frac{\hbar}{m\omega}}a' \\
\vec{r}_i &= \sqrt{\frac{\hbar}{m\omega}}\vec{r'}_i, \\
\Longrightarrow \Delta\vec{r}_{ij} &= \sqrt{\frac{\hbar}{m\omega}}\Delta\vec{r'}_{ij}, \\
\vec{\nabla}_i &= \sqrt{\frac{m\omega}{\hbar}}\vec{\nabla'}_i, \\
\mathbf{H} &= \hbar\omega\mathbf{H}'.
\end{align}
De merka størrelsene $\vec{r'}_i$, $\vec{\nabla'}_i$,  $\Delta\vec{r'}_{ij}$ og $\mathbf{H}'$ er dimensjonsløse, og den merka Hamilton-operatoren tar formen
\begin{equation}
\mathbf{H'}[\vec{R'}] = \sum\limits_i^N \frac{1}{2}\left( U'[\vec{r'}_i] - \vec{\nabla'}_i^2\right) + \sum\limits_i^N\sum\limits_{j > i}^N V'[\Delta{r'}_{ij}]
\end{equation}
med
\begin{align}
U'[\vec{r'}_i] &= x_i^2+y_i^2+\lambda^2z_i^2, \\
V'[\Delta{r'}_{ij}] &=\begin{cases} \infty & \text{for $\Delta{r'}_{ij} < a'$} \\
0 & \text{for $\Delta{r'}_{ij} \geq a'$} \end{cases}.
\end{align}
Vi ser heretter bort ifra merkene og behandler de dimensjonsløse ligningene videre.

Sett nå en prøvefunksjon på formen
\begin{equation}
\Psi[\vec{R}] = \prod\limits_i^N g[\vec{r}_i] \prod\limits_j^N\prod\limits_{k > j}^N f[\vec{r}_j,\vec{r}_k]
\end{equation}
med
\begin{align}
g[\vec{r}_i] &= \epsilon^{-\alpha\left(x_i^2+y_i^2+ \beta z_i^2\right)}, \\
f[\vec{r}_i,\vec{r}_j] &= f[\Delta{r}_{ij}] = \begin{cases} 0 & \text{for $\Delta{r}_{ij} < a$} \\
1-\frac{a}{\Delta{r}_{ij}} & \text{for $\Delta{r}_{ij} \geq a$} \end{cases}.
\end{align}
Her er $\alpha$ og $\beta$ variabler til rådighet i jakta på grunnenergien.

Etter variasjonsmetoden er grunnenergien styrt av
\begin{equation}
E_G \leq \ev{\mathbf{H}}{\Psi}
\end{equation}
for enhver tilstand $\ket{\Psi}$. I posisjonsbasis kan skranken skrives
\begin{equation}
\ev{\mathbf{H}}{\Psi} = \idotsint \Psi^*[\vec{R}]\mathbf{H}[\vec{R}]\Psi[\vec{R}]\delta^{3N}R = \idotsint \abs{\Psi[\vec{R}]}^2\frac{\mathbf{H}[\vec{R}]\Psi[\vec{R}]}{\Psi[\vec{R}]} \delta^{3N}R, \nonumber
\end{equation}
så vi kan altså innføre en lokal energi
\begin{equation}
E_L[\vec{R}] = \frac{\mathbf{H}[\vec{R}]\Psi[\vec{R}]}{\Psi[\vec{R}]}
\end{equation}
og betrakte skranken $\bra{\Psi}\mathbf{H}\ket{\Psi}$ som forventningsverdien til denne lokale energien ved en sjansfordeling $\abs{\Psi[\vec{R}]}^2$, altså
\begin{equation}
E_G \leq \idotsint \abs{\Psi[\vec{R}]}^2E_L[\vec{R}] \delta^{3N}R = \ev{E_L}.
\end{equation}
Det er altså nyttig å se nærmere på $E_L$ og prøve å finne et analytisk uttrykk for denne.

Potergileddene $U$ og $V$ i Hamilton-operatoren gir direkte bidrag til $E_L$, så det er kinergileddet, nærmere bestemt den dobbeltderiverte $\vec{\nabla}_i^2\Psi$ som må beregnes nærmere. Skriv
\begin{equation}
\Psi[\vec{R}] = G[\vec{R}]F[\vec{R}]
\end{equation}
med
\begin{align}
G[\vec{R}] &= \prod\limits_i^N g[\vec{r}_i] \\
F[\vec{R}] &= \prod\limits_j^N\prod\limits_{k > j}^N f[\Delta{r}_{jk}] 
\end{align}
og merk at
\begin{equation}
\vec{\nabla}_i^2\Psi = \vec{\nabla}_i\cdot\left(\vec{\nabla}_i G F + G \vec{\nabla}_i F\right) = \vec{\nabla}_i^2 G F + 2\vec{\nabla}_i G \cdot \vec{\nabla}_i F + G \vec{\nabla}_i^2 F. \label{Lap_Psi_1} 
\end{equation}
Vi må derfor finne de deriverte av henholdsvis $G$ og $F$ for å få et uttrykk for $E_L$. De deriverte av $G$ er 
\begin{align}
\vec{\nabla}_i G &= \vec{\nabla}_i g[\vec{r}_i] \prod\limits_{j \neq i}^N g[\vec{r_j}] \\
\Longrightarrow \vec{\nabla}_i^2 G &= \vec{\nabla}_i^2 g[\vec{r}_i] \prod\limits_{j \neq i}^N g[\vec{r_j}],
\end{align}
og ettersom
\begin{align}
\vec{\nabla}_i g[\vec{r}_i] &= -2\alpha\Big(\vec{r}_i+(\beta-1)\vec{z}_i\Big)g[\vec{r}_i] \\
\Longrightarrow \vec{\nabla}_i^2 g[\vec{r}_i]
&= -2\alpha\left(\Big(D+(\beta-1)\Big)g[\vec{r}_i]+\Big(\vec{r}_i+(\beta-1)\vec{z}_i\Big)\cdot\vec{\nabla}_i g[\vec{r}_i]\right) \nonumber\\
\Longrightarrow \vec{\nabla}_i^2 g[\vec{r}_i]
&= 2\alpha\left(2\alpha\Big(\vec{r}_i+(\beta-1)\vec{z}_i\Big)^2-\Big(D+(\beta-1)\Big)\right)g[\vec{r}_i]
\end{align}
så blir disse ganske enkelt
\begin{align}
&\vec{\nabla}_i G = -2\alpha\Big(\vec{r}_i+(\beta-1)\vec{z}_i\Big)G, \label{grad_G}\\
&\vec{\nabla}_i^2G = 2\alpha\left(2\alpha\Big(\vec{r}_i+(\beta-1)\vec{z}_i\Big)^2-\Big(D+(\beta-1)\Big)\right)G. \label{Lap_G}
\end{align}
For å finne de deriverte av $F$ skriver vi om funksjonen på eksponentialform,
\begin{equation}
F[\vec{R}] = \epsilon^{\sum\limits_{j}^N\sum\limits_{k > j}^N \ln f[\Delta{r}_{jk}]},
\end{equation}
og ser at
\begin{align}
\vec{\nabla}_i F &= \sum\limits_{j \neq i}^N \left(\frac{\vec{\nabla}_i f[\Delta{r}_{ij}]}{f[\Delta{r}_{ij}]}\right) F \\
\Longrightarrow \vec{\nabla}_i^2 F &= \sum\limits_{j \neq i}^N \left(\vec{\nabla}_i\cdot\left[\frac{\vec{\nabla}_i f[\Delta{r}_{ij}]}{f[\Delta{r}_{ij}]}\right]\right) F + \sum\limits_{j \neq i}^N \left(\frac{\vec{\nabla}_i f[\Delta{r}_{ij}]}{f[\Delta{r}_{ij}]}\right)\cdot \vec{\nabla}_i F \nonumber\\
&= \sum\limits_{j \neq i}^N \left(\frac{\vec{\nabla}_i^2 f[\Delta{r}_{ij}]}{f[\Delta{r}_{ij}]} - \frac{\left(\vec{\nabla}_i f[\Delta{r}_{ij}]\right)^2}{f^2[\Delta{r}_{ij}]}\right) F + \sum\limits_{j \neq i}^N \sum\limits_{k \neq i}^N \left(\frac{\vec{\nabla}_i f[\Delta{r}_{ij}] \cdot\vec{\nabla}_i f[\Delta{r}_{ik}]}{f[\Delta{r}_{ij}]f[\Delta{r}_{ik}]}\right) F \nonumber\\
&= \sum\limits_{j \neq i}^N \left(\frac{\vec{\nabla}_i^2 f[\Delta{r}_{ij}]}{f[\Delta{r}_{ij}]} + \sum\limits_{{k \neq i}\atop{k \neq j}}^N \frac{\vec{\nabla}_i f[\Delta{r}_{ij}] \cdot\vec{\nabla}_i f[\Delta{r}_{ik}]}{f[\Delta{r}_{ij}]f[\Delta{r}_{ik}]}\right) F.
\end{align}
De deriverte av $f$ er
\begin{align}
\vec{\nabla}_i f[\Delta{r}_{ij}] &= \frac{a\Delta\vec{r}_{ij}}{\Delta{r}_{ij}^3} \\
\Longrightarrow \vec{\nabla}_i^2 f[\Delta{r}_{ij}]
&= \frac{aD}{\Delta{r}_{ij}^3}-\frac{3a\Delta\vec{r}_{ij}^2}{\Delta{r}_{ij}^5} \nonumber\\
&= \frac{a(D-3)}{\Delta{r}_{ij}^3}
\end{align}
så lenge alle $\Delta{r}_{ij} \geq a$. Merk at
\begin{equation}
\Delta{r}_{ij} f[\Delta{r}_{ij}] = \Delta{r}_{ij}-a
\end{equation}
og dermed blir
\begin{align}
\vec{\nabla}_i F &= \sum\limits_{j \neq i}^N \left(\frac{a\Delta\vec{r}_{ij}}{\Delta{r}_{ij}^2(\Delta{r}_{ij}-a)}\right) F, \label{grad_F}\\
\vec{\nabla}_i^2 F &= \sum\limits_{j \neq i}^N \left(\frac{a(D-3)}{\Delta{r}_{ij}^2(\Delta{r}_{ij}-a)} + \sum\limits_{{k \neq i}\atop{k \neq j}}^N \frac{a^2 \Delta\vec{r}_{ij} \cdot \Delta\vec{r}_{ik}}{\Delta{r}_{ij}^2(\Delta{r}_{ij}-a) \Delta{r}_{ik}^2(\Delta{r}_{ik}-a)}\right) F. \label{Lap_F}
\end{align}

Ligning \eqref{Lap_Psi_1} tar nå formen
\begin{align}
\vec{\nabla}_i^2\Psi &= 2\alpha\left(2\alpha\Big(\vec{r}_i+(\beta-1)\vec{z}_i\Big)^2-\Big(D+(\beta-1)\Big)\right) \Psi \nonumber\\
&- 4\alpha\Big(\vec{r}_i+(\beta-1)\vec{z}_i\Big) \cdot \sum\limits_{j \neq i}^N \left(\frac{a\Delta\vec{r}_{ij}}{\Delta{r}_{ij}^2(\Delta{r}_{ij}-a)}\right) \Psi \nonumber\\
&+ \sum\limits_{j \neq i}^N \left(\frac{a(D-3)}{\Delta{r}_{ij}^2(\Delta{r}_{ij}-a)} + \sum\limits_{{k \neq i}\atop{k \neq j}}^N \frac{a^2 \Delta\vec{r}_{ij} \cdot \Delta\vec{r}_{ik}}{\Delta{r}_{ij}^2(\Delta{r}_{ij}-a) \Delta{r}_{ik}^2(\Delta{r}_{ik}-a)}\right) \Psi \nonumber\\
\Longrightarrow \frac{\vec{\nabla}_i^2\Psi}{\Psi} &= 2\alpha\left(2\alpha\Big(\vec{r}_i+(\beta-1)\vec{z}_i\Big)^2-\Big(D+(\beta-1)\Big)\right) \nonumber\\
&+ \sum\limits_{j \neq i}^N \frac{a(D-3) - 4\alpha\Big(\vec{r}_i+(\beta-1)\vec{z}_i\Big) \cdot a\Delta\vec{r}_{ij}}{\Delta{r}_{ij}^2(\Delta{r}_{ij}-a)} \nonumber\\
&+ \sum\limits_{j \neq i}^N \sum\limits_{{k \neq i}\atop{k \neq j}}^N \frac{a^2 \Delta\vec{r}_{ij} \cdot \Delta\vec{r}_{ik}}{\Delta{r}_{ij}^2(\Delta{r}_{ij}-a) \Delta{r}_{ik}^2(\Delta{r}_{ik}-a)}
\end{align}
og ved å innføre størrelsene
\begin{align}
\vec{q}[\vec{r}_i] &= -4\alpha\Big(\vec{r}_i+(\beta-1)\vec{z}_i\Big), \\
d[\Delta{r}_{ij}] &= \Delta{r}_{ij}^2(\Delta{r}_{ij}-a), \\
\vec{s}[\Delta\vec{r}_{ij}] &= \frac{a\Delta\vec{r}_{ij}}{2d[\Delta{r}_{ij}]}, \\
\end{align}
følger det at det totale utrykket for den lokale energien blir
\begin{align}
E_L[\vec{R}] &= \alpha N\Big(D+(\beta-1)\Big) + \sum\limits_i^N \frac{1}{2}\left( U[\vec{r}_i] - \frac{1}{4}\vec{q}^2[\vec{r}_i] \right) \nonumber\\
&- \sum\limits_i^N \sum\limits_{j \neq i}^N \left( \frac{a(D-3)}{2d[\Delta{r}_{ij}]} + \vec{q}[\vec{r}_i] \cdot \vec{s}[\Delta\vec{r}_{ij}] \right) - 2\sum\limits_i^N \sum\limits_{j \neq i}^N \sum\limits_{{k \neq i}\atop{k \neq j}}^N \vec{s}[\Delta\vec{r}_{ij}] \cdot \vec{s}[\Delta\vec{r}_{ik}] \nonumber\\
&+ \sum\limits_i^N \sum_{j>i}^N V[\Delta{r}_{ij}].
\end{align}
En annen interessant størrelse for systemet er den såkalte kvantekrafta
\begin{equation}
\vec{Q}[\vec{R}] = \frac{2\vec{\nabla}\Psi[\vec{R}]}{\Psi[\vec{R}]}
\end{equation}
Fra ligningene over følger det også at
\begin{align}
\vec{\nabla}_i\Psi &= \vec{\nabla}_i G F + G \vec{\nabla}_i F \nonumber\\
&= -2\alpha\Big(\vec{r}_i+(\beta-1)\vec{z}_i\Big)\Psi + \sum\limits_{j \neq i}^N \left(\frac{a\Delta\vec{r}_{ij}}{\Delta{r}_{ij}^2(\Delta{r}_{ij}-a)}\right) \Psi \nonumber\\
\Longrightarrow \frac{\vec{\nabla}_i\Psi}{\Psi} &= -2\alpha\Big(\vec{r}_i+(\beta-1)\vec{z}_i\Big) + \sum\limits_{j \neq i}^N \left(\frac{a\Delta\vec{r}_{ij}}{\Delta{r}_{ij}^2(\Delta{r}_{ij}-a)}\right),
\end{align}
og dermed blir uttrykket for kvantekrafta
\begin{equation}
\vec{Q}[\vec{R}] = \sum\limits_i^N \vec{q}[\vec{r}_i] + 4\sum\limits_i^N\sum\limits_{j \neq i}^N \vec{s}[\Delta\vec{r}_{ij}].
\end{equation}
med de innførte størrelsene.
\end{document}